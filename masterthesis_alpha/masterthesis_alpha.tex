\documentclass{report}

\usepackage[ansinew]{inputenc}
\usepackage{booktabs}

\usepackage{graphicx}

\usepackage{amssymb} % for using \mathbb{R}

\usepackage{amsmath}

\usepackage{setspace} % for 1.5 line spacing
\onehalfspacing % for 1.5 line spacing

\usepackage{cite}

\usepackage{url} % for using urls

\usepackage{titletoc}% for showing chapters in TOC http://ctan.org/pkg/titletoc
\titlecontents*{chapter}% <section-type>
  [0pt]% <left>
  {}% <above-code>
  {\bfseries\chaptername\ \thecontentslabel\quad}% <numbered-entry-format>
  {}% <numberless-entry-format>
  {\bfseries\hfill\contentspage}% <filler-page-format>


\begin{document}

\graphicspath{{./figures/}} %\graphicspath{{subdir1/}{subdir2/}{subdir3/}...{subdirn/}}


\begin{titlepage}
\centering
\textsc{\Large \mbox{\centerline{Potsdam Institute For Climate Impact Research}}
	\\[1.5ex] University of Potsdam}

\vspace*{4.2cm}

\rule{\textwidth}{1pt}\\[0.5cm]
{\huge \bfseries
  %Introductory phase report \\[0.3ex]
  Introductory phase report\\[1.5ex]
  \Large Optimal adjustment of the global trade system to local network disruption}\\[0.5cm]
\rule{\textwidth}{1pt}

\vspace*{2cm}

\begin{Large}
\begin{tabular}{ll}
Name: & Sebastian Klipp\\	
Matriculation number: & 779142 \\      
Period: & 1.5.15 - 31.9.15\\
E-Mail:   & sklipp@uni-potsdam.de\\	      
Supervisor: & Leonie Wenz \\
Examiner    & Prof. Dr. Anders Levermann
\end{tabular}
\end{Large}

\vspace*{0.8cm}



\end{titlepage}


\tableofcontents

\newpage

%% Alles was mit EORA zu tun hat: Nicht in den Report bringen! Das ist meine Masterarbeit.
%% Hier quasi die Vorarbeiten einbringen, generelles Verst�ndnis der Methoden und Anwendungsbeispiele und daraus erhaltene Erkenntnisse (testworlds)

\chapter{Introduction}

\section{Introduction}
% refer to catastrophes / extreme events
% Trade networks under perturbations
% MRIOTs

Forecasts for the next century predict an increase of global extreme events like droughts, floods or tornados. The report of the 
\textit{Intergovernmental Panel on Climate Change (IPCC)}\footnote{Intergovernmental Panel on Climate Change (IPCC). Managing the Risks of Extreme 
Events and Disasters to Advance Climate Change Adaptation. A Special Report of Working Groups I and II of the Intergovernmental Panel on Climate 
Change (SREX). [Field, C.B. et al.]. Cambridge, UK and New York, NY, USA : Cambridge University Press, 2012} confirms the relevance of extreme events. 
Nature catastrophies cause destruction and misery in the affected region, inter alia whole economic sectors are struck leading to a reduction in 
production capabilites. As a result, economic sectors connected by the global trade network to the affected one could also be influenced causing the 
need to adapt their production, possibly leading to second order effects on further economic sectors. As a consequence of this effect a local 
perturbation spreads into the global trade network and therefore affects economic sectors all over the world - potentially even those far away from 
the source of catastrophy.\\
A representation of the global trade network can be obtained by using \textit{multi regional input output tables (MRIOT)}, which contain information 
about all significant intermediate monetary flows between economic sectors of the world's regions and their final demands (consumption). A 
comprehensive MRIOT data set is the EORA data set\footnote{Lenzen M, Kanemoto K; Moran D, and Geschke A (2012) Mapping the structure of the world 
economy, Environmental Science \& Technology 46(15) pp 8374-8381}, which is used in this work. \\
Imposing fundamental rules upon the global trade network, namely the production output balance equation, the supply of required inputs and the
necessity of satisfying the final demands of each region, defines the space of the network's reaction behaviours in answer to 
perturbations. Satisfying these equations, each economic sector adapts its input/output flows and produciton ratio, while the latter is kept 
as close to its initial (optimal) value as possible. Temporary over- or underproductions lead to additional global costs of adaptation. These rules 
lead to a linear optimisation problem (see chapter \ref{chapter:LP}), which is solved by using the simplex method (see chapter 
\ref{chapter:simplex}).


\chapter{Theory}





\section{Simplex method}
% simplex algorithm
% simplex vs. other algorithms
% degenerated solutions

\label{chapter:simplex}

\subsection{Mathematical derivation}

The simplex optimisation method\footnote{This mathematical derivation is based on the script ``Lineare Optimierung'' of Prof. Dr. Bernhard Schmitt, 
University of Marburg} is applied to linear programs
\begin{gather}
 \underset{x}{\min} \{c^T x: x\in X\}, \\
  X := \{x \in \mathbb{R}^n: Ax = b, x \ge 0\},
\end{gather}
with $A \in \mathbb{R}^{m\times n}, b \in \mathbb{R}^m$ and $\text{rank}(A)=m < n$. The solution space $X$ is restricted by hyperplanes $Ax=b$ and 
$x\ge 0$, forming the shape of a convex polyhedron. For an under-determined system of equations ($m<n$) a special (not necessarily optimal) basic 
solution $\bar{x} \in X$ exists with $m$ active components $\bar{x}_J \ge 0$, assigned by inidices $\{j\} = J \subseteq N := \{1,...,n\}$. The 
remaining $n-m$ inactive components equal zero and are identified by indices $\{k\} = K = N \setminus J$, so that $\bar{x} = (\bar{x}_J,\bar{x}_K)$. 
Analogously, the constraint matrix $A$ can be 
seperated into a submatrix $A_J=(a_{j1},...,a_{jm}) \in \mathbb{R}^{m\times m}$, containing the columns corresponding to the active components, and a 
matrix $A_K=(a_{k1},...,a_{km}) \in \mathbb{R}^{m\times n-m}$ containing the inactive columns of 
$A$, resulting in $A=(A_J, A_K)$. Solution $\bar{x}$ is called the basis solution with basis $A_J$. Using the above partitions yields
\begin{align}
 b &= A\bar{x} = A_J \bar{x}_J + A_{K} \bar{x}_K\;,\\
 \xrightarrow{\bar{x}_K = 0} b &= A_J \bar{x}_J \;. \label{eq:basis}
\end{align}
Moreover, with knowledge of a special solution $\bar{x} = (\bar{x}_J, \bar{x}_K)$, other points $x=(x_J,x_K)\in X$ ($x_K$ not necessarily zero) on 
the edge of the polyhedron can be created, using Eq(\ref{eq:basis})
\begin{align}
 b &= Ax = A_J x_J + A_K x_K \\
 \leftrightarrow x_J &= A_J^{-1} b - A_J^{-1} A_K x_K \\
                     &= \bar{x}_J - A_J^{-1} A_K x_K\;.
\end{align}
Substituting $x_K$ by the parameter $\lambda_K$ leads to
\begin{align}
 x = \begin{pmatrix} x_J \\ x_K \end{pmatrix} = \begin{pmatrix} \bar{x}_J \\ 0_K \end{pmatrix} + 
\begin{pmatrix} - A_J^{-1} A_K \\ I_{n-m} \end{pmatrix} \lambda_K = \bar{x} - W_K \lambda_K \ge 0\,,
\end{align}
with 
\begin{align}
 \begin{pmatrix} - A_J^{-1} A_K \\ I_{n-m} \end{pmatrix} = \begin{pmatrix} W_K^{(J)} \\ W_K^{(K)} 
\end{pmatrix} = W_K \in \mathbb{R}^{n \times (n-m)}\;.
\end{align}
By increasing the value of a component $x_l \in x, l\in K$ one moves along a ray inside the solution space $X$, originating in $\bar{x}$. All rays 
for each $l\in K$ create a cone with $\bar{x}$ being the peak and a corner of the convex polyhedron $X$. One elementary ray is 
\begin{align}
 x(\lambda_l) := \bar{x} - \lambda_l w_l \leftrightarrow \begin{cases} x_J(\lambda_l) = \bar{x}_J - \lambda_l A_J^{-1} a_l \\ 
x_K(\lambda_l) = \lambda_l \delta_{kl}, k\in K \end{cases}\;,
\end{align}
with $w_l$ being the l-th column of $W_K$, $a_l$ the l-th column of $A_K$ and $\delta_{kl}$ being the 
Kronecker Delta.

The fundamental concept of the Simplex method is to start from one basic solution $\bar{x}$ in the corner of the polyhedron and move along one 
selected ray $x(\lambda_l)$ towards another corner, reducing the target function meanwhile and repeating this procedure until the optimal solution 
corner $x^*$ is found. To figure out which ray originating in $\bar{x}$ reduces the target function, we consider the cost function
\begin{align}
 c^T x &= c_J^T x_J + c_K^T x_K = c_J^T (\bar{x}_J - A_J^{-1} A_K x_K) + c_K^T x_K\\
       &= \underbrace{c_J^T \bar{x}_J}_{c^T \bar{x}} + \underbrace{(x_K^T - c_J^T A_J^{-1} A_K)}_{\gamma_K^T} x_K = c^T \bar{x} + \gamma_K^T 
x_K\;.
\end{align}
The cost function in the area of $\bar{x}$ changes by increasing the non-basis variables $x_K=\lambda_K$, weighted by the vector of reduced costs 
$\gamma_K$. Moving along a ray $x(\lambda_l)$ with negative reduced costs 
$\gamma_l$  leads towards the minimal solution of the linear program. The minimum is reached when each element of $\gamma_K$ is non-negative 
$\gamma_l\ge 0\;; \forall l$. In this case the corresponding special basic corner solution $\bar{x}$ is identified as the optimal solution $x^*$. 
While moving along the ray it is important to ensure that every component of $x$ remains inside the solution area $X$. Consider ray $x(\lambda_l)$ in 
component form
\begin{align}
 x_j(\lambda_l) = \bar{x}_j - \lambda_l w_{jl}\;.
\end{align}
While increasing $\lambda_l \ge 0$ the components $x_j(\lambda_l)$ will increase if $w_{jl} \le 0$. For executing a basis change $\lambda_l$ will be 
set to a value, so that one component $x_p(\lambda_l)$ gets equal to zero and can therefore 
be removed from the active set. For this reason, only those components with $w_{jl} > 0$ are considered. To ensure that all components of $x$ stay 
inside the solution area $X$, $\lambda_l$ is increased until the first component $x_p$ gets zero
\begin{align}
 \lambda_l := \min \{ \frac{\bar{x}_i}{w_{il}} : i\in J, w_{il} >0\} = \frac{\bar{x}_p}{w_{pl}} \ge 0\;.
\end{align}
The index $p$ is added to the inactive set $K$, while variable $l$ is included into the new basis set $J$
\begin{align}
 J &= J \setminus \{p\} \cup \{l\}\\
 K &= N \setminus J \;.
\end{align}
resulting in a new corner solution $\bar{x}=(\bar{x}_J, \bar{x}_K)$ moving the target function $F1$ closer to \cite{Braess} the minimum.\\

afdadfasfdasfasdfas
\subsection{Simplex application scheme}
% Computing time Aufwand 6m^2 (skript S19)



\begin{table}[!h]
\onehalfspacing
 \begin{tabular}{|c|c|}

\hline
Input	&Initial feasible basis $A_J$, $J \subseteq N$, $N := \{1,...,n\}$	\\
\hline
Step 1	&$\bar{x}_J := A_J^{-1} b$, $K := N \setminus J$	\\
\hline
Step 2	&search for $\gamma_l < 0$ out of $\gamma_j := c_j - c_J^T A_J^{-1}a_j; j\in K$	\\
\hline
Step 3	&if $\gamma_j \ge 0\; \forall j \in K$: \textbf{STOP, optimum found} \\
\hline
Step 4	&$w_l^{(J)}:=A_J^{-1}a_l$\\
\hline
Step 5	&if $w_{il} \le 0\; \forall i \in J$: \textbf{STOP, unbounded solution}	\\
\hline
Step 6	&Identify $p \in J$ with: $\frac{\bar{x}_p}{w_{pl}} := \min \{ \frac{\bar{x}_i}{w_{il}} : w_{il} > 0, i\in J \} =  \lambda_l$ \\
\hline
Step 7	&$J:= J \setminus \{p\} \cup \{l\}$; \textbf{jump to step 1} \\
\hline
 
 
 \end{tabular}
 \caption{Simplex algorithm}
  
\end{table}


\chapter{Model setup}

\section{problem specific linear optimisation problem}
% decision for linear program (vs. nonlinear / quadratic program)
% LPG
%  Explanation of constraints
%  formula for varn and eqn
% LPS
%  Explanation of constraints
%  formula for varn and eqn
% goal function F=1.

\label{chapter:LP}

The data of a MRIOT can be interpreted as a network representation with $n_r$ global regions, each containing $n_i$ economic sectors, 
so that $n_{ir}=n_i\cdot n_r$ yields the total amount of regional sectors. Each node $ir \in V:= \{1,...,n_{ir}\}$ represents a regional sector with 
a production ratio $p_{ir}=\frac{X^{new}_{ir}}{X^{old}_{ir}}$, $X^{old}_{ir}$ indicating the initial total production of $ir$ and $X^{new}_{ir}$ the 
adapted total production. The nodes $ir\in V$ and $js\in V$ are connected by $n_{qZ}$ intermediate flows $Z_{ir\rightarrow js}$, weighted with the 
flow ratios $q_{ir\rightarrow js}$. Similarly, the $n_{qY}$ final demand flows $Y_{ir\rightarrow s}$ are weighted with the flow ratios 
$q_{ir\rightarrow s}$. \\
A linear optimisation problem is formulated, using $\{p_{ir}\}$, $\{q_{ir\rightarrow js}\}$ and $\{q_{ir\rightarrow s}\}$ as free (non-leading) 
variables, which are minimised according to the target function $T1(\{p_{ir}\}$, fulfilling the constraints $Ax = b$. The constraint matrix $A$ 
consists of three different type of constraints $C1$, $C2$ and $C3$, which are described in the following. Two different linear problems LP1 and LP2 
are formulated which both will be object of research.



\subsection{Target function}
% decision for linear program (vs. nonlinear / quadratic program)
% absolute value just mention it
% cost vector c^T
\label{eq:targetF}

The optimisation procedure is executed by using the following target function
\begin{align}
 F(\{p_{ir}\}) = \sum_{ir} |p_{ir} - 1| \tag{T1}
\end{align}
The vector of costs $c$ describes the weighted costs for each variable used. Since no flow variables appear in the target function, their 
corresponding costs are set to zero, while each production ratio variable $p_{ir}$ is weighted evenly with a value of one
\begin{align}
 c = (\underbrace{1,...,1}_{n_{ir}}\;\,,\underbrace{0,...,0}_{n_{qY}+n_{qZ}})\;.
\end{align}




\subsection{Final demand constraint}
\label{eq:C1}

\begin{align}
  \sum_{r} q_{ir\rightarrow s}Y_{ir\rightarrow s} = \sum_r Y_{ir\rightarrow s} \;; \forall s; \forall i \tag{C1}\;.
\end{align}
The equation's left-hand side sums up all adaptable final demand flows $q_{ir\rightarrow s}Y_{ir\rightarrow s}$ of a sector $i$ into a region $s$. 
The result has to equal the sum of all final demand flows $Y_{ir\rightarrow s}$ of the initial unperturbed network.\\
Each region receives at most $n_i$ different input goods, therefore the maximum number of final demand constraints (C1) of the linear problem amounts 
to $n_{C1} = n_r \cdot n_i$ constraints.


\subsection{Supply scaling constraint}
\label{eq:C2}

\begin{align}
  \sum_r q_{ir\rightarrow js} Z_{ir\rightarrow js} = p_{js} \sum_r Z_{ir\rightarrow js}  \;; \forall js; \forall i \tag{C2}\;.
\end{align}
The sum of all adaptable intermediate flows $q_{ir\rightarrow js} Z_{ir\rightarrow js}$ of a sector $i$ into a node $js$ has to match the sum of its 
original inputs $Z_{ir\rightarrow js}$, scaled by its adaptable production ratio $p_{js}$. \\
%An adapted production ratio leads to an equally adjusted demand for inputs.\\
At most, each node $js$ requires an input of each type of good, hence the maximum number of supply scaling constraints (C2) of the linear program 
amounts to $n_{C2} = n_{ir} \cdot n_i = n_r \cdot n_i^2$.



\subsection{Production output balance constraint}
\label{eq:C3}

\begin{align}
  p_{ir} X_{ir} &= \sum_{js} q_{ir\rightarrow js} Z_{ir\rightarrow js} + \sum_{s} q_{ir\rightarrow s} Y_{ir\rightarrow s} \;; \forall ir\;. \tag{C3}
\end{align}
The adaptable production $p_{ir} X_{ir}$ of a node $ir$ has to equal the sum of its outgoing adaptable intermediate flows $q_{ir\rightarrow js} 
Z_{ir\rightarrow js}$ plus the sum of its outgoing adaptable final demand flows $q_{ir\rightarrow s} Y_{ir\rightarrow s}$.\\
This constraint has to be fulfilled by each node, therefore the number of production output balance constraints (C3) amounts to $n_{C3} = n_{ir}=n_r 
\cdot n_i$. 



\subsection{Linear problem 1 (LP1): Maximal adaptation}
% LPG
% free variables
% if i*r* drops out
% formula for varn and eqn
The linear problem with maximal adaptation (LP1) is composed of the target function (\ref{eq:targetF}) and the constraints (\ref{eq:C1}) 
(\ref{eq:C2}) (\ref{eq:C3}):
\begin{align}
\hline
 & & \underset{p_{ir}}{\min}\; F &= \sum_{ir} |p_{ir} - 1| \tag{T1} \\
 & s.t. & \sum_{r} q_{ir\rightarrow s}Y_{ir\rightarrow s} &= \sum_r Y_{ir\rightarrow s} \;;  \forall s;\forall i  \tag{C1} \\
 & & \sum_{r} q_{ir\rightarrow js} Z_{ir\rightarrow js} &= p_{js} \sum_{r} Z_{ir\rightarrow js}   \;;  \forall js;\forall i \tag{C2} \\
 & & p_{ir} X_{ir} &= \sum_{js} q_{ir\rightarrow js} Z_{ir\rightarrow js} + \sum_{s} q_{ir\rightarrow s} Y_{ir\rightarrow s} \;; \forall ir 
\tag{C3}\;.\\
\hline \nonumber
\end{align} 
All production ratio variables $p_{ir}$ and flow ratio variables $q_{ir\rightarrow js}, q_{ir\rightarrow s}$ are treated as a free variables 
under the optimisation procedure, so the solution vector $x$ yields
\begin{align}
  x=(\{p_{ir}\}, \{q_{ir\rightarrow js}\}, \{q_{ir\rightarrow s}\}); \forall ir, \forall js, \forall s\;.
\end{align}
The maximal amount of free variables $n_{var}$ consists of the amount of production ratio variables $n_{ir} = n_r \cdot n_i$, the amount of 
intermediate flows $n_{qZ} = n_{ir} \cdot n_{ir} = n_{ir}^2$ plus the amount of final demand flows $n_{qY} = n_r \cdot n_{ir} = n_r^2 \cdot n_i$
\begin{align}
 n_{var} = n_{ir}^2 + n_r^2 \cdot n_i + n_r \cdot n_i\;.
\end{align}
This represents a fully connected network, with a maximal amount of constraints
\begin{align}
 n_{C} = n_{C1} + n_{C2} + n_{C3} =  n_r \cdot n_i^2 + 2\; n_r \cdot n_i\;.
\end{align}
In the case of a drop out of node $i^*r^*$, its production ratio $p_{i^*r^*}$ together with all intermediate input flows from other nodes $js$ into 
$i^*r^*$ 
$q_{js\rightarrow i^*r^*}$  and all outgoing flows $q_{i^*r^*\rightarrow js}, q_{i^*r^*\rightarrow s}$ are set to zero and removed from the set of 
free variables $x$:
\begin{align}
 p_{i^*r^*}&=0\;,\\
 q_{i^*r^*\rightarrow js}&=0; \forall js\;,\\
 q_{i^*r^*\rightarrow s}&=0; \forall s\;,\\
 q_{js\rightarrow i^*r^*}&=0; \forall js\;.
\end{align}





\subsection{Linear problem 2 (LP2): Reduced adaptation}
% LPS
% assumptions
% free variables
% if i*r* drops out
% formula for varn and eqn

By reducing the amount of free variables $x$ of the optimisation problem by transforming some of them into leading variables (which are dependent on 
the free variables), one can influence the resulting network properties and save computing time.\\
In the case of the linear problem 2 (LP2) the focus of adaptation is set on flows transporting goods $i^*$, which shall remain freely adaptible and 
therfore part of the free variables $x$, while all other flows with $i\neq i^*$ are omitted
\begin{align}
 x=(\{p_{ir}\}, \{q_{i^*r\rightarrow js}\}, \{q_{i^*r\rightarrow s}\}); \forall i; \forall r; \forall js; \forall s\;.
\end{align}
The flows with $i\neq i^*$ are determined by predefined equations
\begin{align}
\label{eq:preeq}
 q_{ir\rightarrow js} &= p_{js} \;; \forall i\neq i^*; \forall r; \forall js\;,\\     
 q_{ir\rightarrow s}  &= 1 \;; \forall i\neq i^*; \forall r; \forall s \;.
\end{align}
This formulation of a linear problem keeps all final demand flows $Y_{ir\rightarrow s}$ with goods $i\neq i^*$ constant under perturbation. In 
addition, all intermediate input flows $Z_{ir\rightarrow js}$ with $i\neq i^*$ are assumed to be equal to the production ratio $p_{js}$ of the input 
node $js$. This means that a change of production $p_{js}$ and therefore a change of inputs $Z_{ir\rightarrow js}$ is evenly distributed over all 
suppliers of node $js$.\\
Using the above changes, the linear problem 1 (LP1) transforms into the new linear problem 2 (LP2)
\begin{align}
\hline
 & & \underset{p_{ir}}{\min}\; F &= \sum_{ir} |p_{ir} - 1| \tag{T1} \\
 & s.t. & \sum_{r} q_{i^*r\rightarrow s}Y_{i^*r\rightarrow s} &= \sum_r Y_{i^*r\rightarrow s} \;; \forall s  \tag{C1} \\
 & & \sum_{r} q_{i^*r\rightarrow js} Z_{i^*r\rightarrow js} &= p_{js} \sum_{r} Z_{i^*r\rightarrow js}\;; \forall js  
\tag{C2} \\
 & &p_{i^*r} X_{i^*r} &= \sum_{js} q_{i^*r\rightarrow js} Z_{i^*r\rightarrow js} + \sum_{s} q_{i^*r\rightarrow s} Y_{i^*r\rightarrow s} 
\;; \forall r \tag{C3a} \\
 & & p_{ir} X_{ir} &= \sum_{js} p_{js} Z_{ir\rightarrow js} + \sum_s Y_{ir\rightarrow s} \;; \forall i\neq i^*;\; \forall r \tag{C3b}\\
\hline \nonumber
\end{align}
The final demand constraints (C1) only need to be formulated for flows with good $i^*$, because the predefined equations (\ref{eq:preeq}) determine 
all final demand flows with $i\neq i^*$, making the corresponding constraints dispensable. The maximal amount of final demand constraints (C1) 
reduces 
to $n_{C1} = n_r$. Similarly the supply scaling constraint (C2) is reduced to those equations using flows with good $i^*$. All other intermediate 
flows are determined, leading to omittance of the corresponding equations and to a maximal amount of supply scaling constraints of $n_{C2} = n_{ir} 
= n_r \cdot n_i$. The production output balance constraint (C3) is split into two constraints (C3a) and (C3b). The first is valid for nodes which 
produce the good $i^*$ and keeps the form of LP1 with adaptability of intermediate flows as well as final demand flows. Contrarily, for nodes with 
$i\neq i^*$ constraint (C3b) shows with a reduced adaptability, because both flow variables are substituted by the predefined equations, only leaving 
the production ratio $p_{ir}$ as a free variable. Still there is exactly one production output balance constraint for each node $n_{C3} = n_r \cdot 
n_i$. Altogether, the maximal amount of constraints for LP2 adds up to
\begin{align}
 n_{C} = n_{C1} + n_{C2} + n_{C3} = n_r + 2\; n_r \cdot n_i\;.
\end{align}
The maximal amount of intermediate flow variables is reduced to $n_{qZ} = n_{ir}\cdot n_r = n_r^2 \cdot n_i$, because each node gets one input with 
$i^*$ from each regional sector $i^*r$. Moreover, the amount of final demand flow free variables equals $n_{qY} = n_r^2$. Adding the amount of 
production ratios $n_{ir}=n_r \cdot n_i$, the maximal amount of variables for LP2 amounts to
\begin{align}
 n_{var} = n_r^2 \cdot n_i + n_r^2 + n_r \cdot n_i\;.
\end{align}
Analogously to LP1, the drop out of node $i^*r^*$ leads to the removal of following variables out of the set of $x$
\begin{align}
 p_{i^*r^*}&=0\;,\\
 q_{i^*r^*\rightarrow js}&=0; \forall js\;,\\
 q_{i^*r^*\rightarrow s}&=0; \forall s\;,\\
 q_{js\rightarrow i^*r^*}&=0; \forall js\;.
\end{align}


\section{Analysis of the EORA network}


\section{Aggregated network}


\chapter{Results}

\section{statistics}

\subsection{F(ir)}

\subsection{comp\_ir(ir)}

\section{comparison LPS/LPG}

\section{absorption potential}

\section{linear response}

\section{phase transition}

\section{case studies incl. forward/backward effects}

\subsection{Japan machinery drops out}

\subsection{other forward effect example}

\section{??? time evolution ???}




\chapter{Final}

\section{discussion}

\section{Ausblick}



\section{Appendix}

\bibliography{sources123}{}
\bibliographystyle{unsrt}

gsadgsdfgsfd

cnbxvcb






\end{document}